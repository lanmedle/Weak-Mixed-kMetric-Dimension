\documentclass{article}
\usepackage[utf8]{inputenc}
\usepackage[slovene]{babel}
\usepackage{amsmath}
\usepackage{amsthm}
\usepackage[a4paper, margin = 3cm]{geometry}
\usepackage{amsfonts}
\usepackage[shortlabels]{enumitem}

\theoremstyle{definition}
\newtheorem{definition}{Def.}
\newtheorem{theorem}{Izrek}
\newtheorem{lemma}{Lema}
\newtheorem{corollary}{Posledica}
\newtheorem{example}{Primer}
\newtheorem{remark}{Opomba}
\newtheorem{proposition}{Trditev}

\author{Lan Medle, Nina Švigelj}
\title{Šibke mešane $k$-metrične dimenzije}

\begin{document}
\maketitle

\section{Opis problema}
Pravimo, da vozlišče $s$ razreši par vozlišč $x,y$ v grafu $G$, če velja: $$d(s,x) \neq d(s,y).$$
Množica vozlišč S je razrešitev za vsak graf $G$, če vsak par vozlišč $x,y$ v $G$ razrešuje neko vozlišče $s \in S.$

(Vozliščna) metrična dimenzija povezanega grafa $G$, označena kot $dim(G)$, je velikost najmanjše množice $S\subseteq V(G)$, ki razlikuje vse pare vozlišč v $G$.

Midva bova preučevala mešano metrično dimenzijo, kjer v grafu razrešujemo tako robove kot točke, kar pomeni, da želimo razlikovati vsak par točk, vsak par robov in vsako točko od vsakega roba. Označena z $mdim(G)$, je velikost najmanjše množice $S \subseteq V(G)$, ki razlikuje vse pare točk in robov.

Naj bo $S \subseteq V(G)$ in $a,b \in V(G) \cup E(G)$. Definiramo $\Delta_S (a,b)$ kot vsoto razlik razdalj od $a$ in $b$ do vsakega vozlišča iz $S$, torej $$\Delta_S(a,b) = \sum_{s\in S} |d(s,a) - d(s,b)|.$$

\noindent
Označimo $\Delta_{V(G)} (a,b) = \Delta(a,b)$.

Šibka mešana $k$-metrična dimenzija $wmdim_k(G)$ grafa $G$ je definirana kot moč najmanjše mnoice vozlišč $S$, za katero velja, da za vsak par vozlišč ali povezav $a,b \in V(G) \cup E(G)$ velja $$\Delta_S(a,b) \ge k.$$
\noindent
Maksimalno vrednost $k$ za katero je definirana šibka mešana $k-$metrična dimenzija z $\kappa''(G).$


\vspace{0.3cm}
\noindent
Najina naloga je napisati CLP program za različico naslova in v programu Sage napisati manjše podprograme, ki nama bodo pomagali odgovoriti na naslednja vprašanja:
\begin{enumerate}
    \item Določite $\kappa''(G)$ in $wmdim_k(G)$ za cikle, polne grafe, dvodelne polne grafe, hiperkocke in kartezične produkte ciklov, ter poskusite na podlagi izračunov uganiti možne formule.
    \item Poskusite določiti grafe $G$ za katere je $wmdim_k(G)$ majhen, recimo 1,2 ali 3. Pravtako določite tudi grafe za katere je $wmdim_k(G)$ velik, recimo $n$, $n-1$ ali $n-2$, kjer je $n$ \textbf{red} grafa $G$ (\textit{kar je enako številu vozlišč v grafu}).
\end{enumerate}
Za manjše grafe poiščite $wmdim_k(G)$ z uporabo sistematičnega iskanja, za večje pa s stohastičnim iskanjem. Sestavite poročilo o svojih rezultatih.


\section{Potek dela}
\begin{enumerate}
    \item Konstrukcija CLP za različne tipe grafov.
    \item S pomočjo konstruiranih CLP poiščeva $\kappa''(G)$ in $wmdim_k(G)$ za specifične primere ter poskusiva uganiti splošno formulo za $\kappa''(G)$ in $wmdim_k(G)$.
    \item Poskusila bova določiti grafe $G$ za katere je $wmdim_k(G)$ majhen oz. velik.
    \item Napisala bova poročilo.
\end{enumerate}


% \subsection*{Slovar tujk}
% \textbf{resolve} razrešiti


\end{document}